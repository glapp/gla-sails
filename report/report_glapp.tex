\documentclass{seal_thesis}

\usepackage[
  top=1.25in,
  bottom=1.25in,
  left=1.25in,
  right=1.25in
]{geometry}

\thesisType{Master Project Report}
\date{\today}
\title{GLAPP}
\subtitle{}
\author{
Fabio Isler \textmd{(09-115-965)} \\
Man Wai Li \textmd{(14-705-156)} \\
Dinesh Pothineni \textmd{(14-707-988)} \\
Riccardo Patane \textmd{(XX-XXX-XXX)}}
\home{} % Geburtsort
\country{}
\prof{Prof. Dr. Harald C. Gall}
\assistent{Dr. Philipp Leitner}
\email{}

\begin{document}
\maketitle

\abstract
Abstract


%%%%%%%%%%%%%%%%%%%%%%%%%%%%%%%%%%%%%%%%%%%%%%%%%%%%%%%%%
% NEW CHAPTER																						%
% Remember to always use a new line for a new sentence! %
%%%%%%%%%%%%%%%%%%%%%%%%%%%%%%%%%%%%%%%%%%%%%%%%%%%%%%%%%
\chapter{4 PAGES: Introduction}
\todo{DONE BY: Fabio}

Cloud computing has created a paradigm shift in the last few years, by making infrastructure available at lower costs and with higher efficiency of operations.
These solutions are increasingly being adopted by enterprises and developers, as they can provision huge amount of resources to scale on-demand in order to meet their business needs.
In cloud computing, resources such as CPU processing time, disk space, or networking capabilities, are rented and released as a service.
Today, the most important model for delivering on the cloud promise is the Infrastructure-as-a-Service (IaaS) paradigm.
In IaaS, virtual computing resources are acquired and released on demand, either via an Application Programming Interface (API) or web interface.
Along with great flexibility of being able to get new resources on demand and pay for what you use, new problems arise.
Selecting a cloud service provider can often be a quite challenging decision for the developer or a company, so is being able to monitor and evaluate these infrastructure resources on a regular basis.
With so much of variance in cost and performance, it is imperative that one would look for a reliable deployment, monitoring solutions to strike a balance with application requirements.
Furthermore, the complexity and skill required to manage multiple layers of application, data, middleware and operating systems can .

Understanding run time performance, behavior of various application components in realtime can enable us to take advantage of arbitrage opportunities that exist between different machines/regions or even other cloud providers.
For example, an application can take advantage by moving closer to its users based on timezones or traffic to improve response time.
Currently there is no flexibility to move freely between various cloud providers without great development effort and cost, however such an ability to move freely between providers enables us to benefit from cost and performance differences.
We propose a cloud middleware that can take not only take care of application deployment to cloud, but also constantly monitor and trigger such necessary adaptations to benefit from these opportunities.
Ideally this middleware also enables developers to specify their intended goals in terms of high level policies to govern the application behavior.
The middleware can then break down these policies into low level objectives, in order to trigger adaptations by changing the state of application when required.

%%%%%%%%%%%%%%%%%%%%%%%%%%%%%%%%%%%%%%%%%%%%%%%%%%%%%%%%%
% NEW SECTION																						%
%%%%%%%%%%%%%%%%%%%%%%%%%%%%%%%%%%%%%%%%%%%%%%%%%%%%%%%%%
\section{Global Living Cloud Applications}

The aim of this project is to develop a middleware for what we call ''global living cloud applications'' (GLAs).
In a nutshell, GLAs are a bio-inspired notion of cloud-native applications.
GLAs ''live'' in the cloud, and are able to migrate between data centers and cloud providers automatically, based on changes in cost and performance of cloud offerings, changes in customer behavior or requirements, or other factors.

\todo{''What I would really do is invest more thought and discussions on the conceptual model behind GLAs - I think there is something pretty cool here, but it does not seem very well-developed to me.''}


The bio-inspired terminology applies for the different levels of components of a GLA:

\begin{itemize}
	\item Cell: A cell is the lowest-level component of a GLA, consisting of the actual processes \todo{''I wonder whether it makes sense to equate cells to "processes". Is that really the same?''}
	\item Organ: An organ consists of one or more cells of the same type \todo{''For organs, I understand that organs typically also have cells of different type, i.e., an injected load balancer. I would try to make the model clear in this regard. For instance, it may make sense to distinguish between cells that are user-defined and those that come from your middleware.''}
	\item GLA: The GLA itself is a collection of organs that form the whole application
\end{itemize}

In order to manage these GLAs, we introduce a middleware called GLAPP (Global Living Application Platform).
It allows a developer to deploy multiple GLAs on whatever cloud he/she has access to and sets a centralized mechanism in place to constantly monitor and manage all the GLAs.
The middleware supports heterogeneous environment a GLA can live and move across different providers, regions, instance types, etc.
GLAPP is an open source middleware to be used on a private/individual basis.
This means that in order to be able to deploy a GLA through GLAPP, it is first required to install the GLAPP platform on the own infrastructure.
But no matter where the platform is installed, all GLAs will live in the cloud.


%%%%%%%%%%%%%%%%%%%%%%%%%%%%%%%%%%%%%%%%%%%%%%%%%%%%%%%%%
% NEW CHAPTER																						%
% Remember to always use a new line for a new sentence! %
%%%%%%%%%%%%%%%%%%%%%%%%%%%%%%%%%%%%%%%%%%%%%%%%%%%%%%%%%
\chapter{6 PAGES: Background \& Architecture}

%%%%%%%%%%%%%%%%%%%%%%%%%%%%%%%%%%%%%%%%%%%%%%%%%%%%%%%%%
% NEW SECTION																						%
%%%%%%%%%%%%%%%%%%%%%%%%%%%%%%%%%%%%%%%%%%%%%%%%%%%%%%%%%
\section{Basic Design Decisions}
\todo{IN CHARGE: Adrian}

\subsection{Main Components: Provisioning Backend, Frontend, MAPE}
\todo{Describe the design decisions of the main components}

\subsection{Deployment: Containerization with Docker}
\subsubsection{Containerization vs. other virtualization methods:}
\todo{Explain the advantage of containerization compared to e.g. virtual machines}
\subsubsection{Docker vs. other containerization implementations:}
\todo{Explain the advantage of Docker compared to e.g. OpenVZ}

\subsection{Orchestration: Docker Swarm}
\todo{Explain the advantage of Docker Swarm compared to e.g. Kubernetes}

\subsection{Rule-Based Adaptations vs. Markov Decision Process}
\todo{DONE BY: Adrian / Riccardo}
\todo{Explain the thought process behind this decision}


%%%%%%%%%%%%%%%%%%%%%%%%%%%%%%%%%%%%%%%%%%%%%%%%%%%%%%%%%
% NEW SECTION																						%
%%%%%%%%%%%%%%%%%%%%%%%%%%%%%%%%%%%%%%%%%%%%%%%%%%%%%%%%%
\section{Implementation Decisions}
\subsection{Backend: Node.js}
\todo{DONE BY: Fabio}
\todo{Explain the choice and alternatives}

\subsection{GUI: AngularJS}
\todo{DONE BY: Dinesh}
\todo{Explain the choice and alternatives}

\subsection{MAPE: Java}
\todo{DONE BY: Adrian / Riccardo}
\todo{Explain the choice and alternatives}
Java is a widely used programming language with good interoperability and decent support by other system or libraries used in this project.
MAPE is a key component of the GLAPP platform.
One of the important aspect of MAPE is the ability to easily interface with other components and system inside or outside GLAPP platform.
Another consideration is the availability of third party libraries that GLAPP can leverage to perform common computation using well-known algorithm such as various planning and learning algorithms used in Markov Decision Process (MDP).

A set of comprehensive functionality is provided through Java API and third party libraries.
For instance, though Java standard API, robust interfacing with SAILS backend monitoring system through HTTP connection is readily available.
In addition, GSON library \todo{add reference} provides easy API to create and manipulate JSON objects that is used for data exchange between MAPE and SAILS backend as well as between MAPE and monitoring system.
Most importantly, BURLAP library \todo{add reference} provides not only a set of planning and learning algorithm for Markov Decision Process in reinforcement learning, but also a framework for further extending the processing capability through custom implementation of various components including learning algorithm and approximation function.
Availability of these functionalities from built-in and third party library makes Java a compelling language in developing MAPE component.


%%%%%%%%%%%%%%%%%%%%%%%%%%%%%%%%%%%%%%%%%%%%%%%%%%%%%%%%%
% NEW SECTION																						%
%%%%%%%%%%%%%%%%%%%%%%%%%%%%%%%%%%%%%%%%%%%%%%%%%%%%%%%%%
\section{Architecture}
The platform consists of 3 different parts/blocks: a front end, a provisioning component and a control loop.
The front end provides an interface for developer to interact with the middleware to deploy and manage his/her GLAs.
The provisioning component provides the management functionalities of the middleware including cloud infrastructure management, application deployment and access to the application status information.
Lastly the control loop is the component responsible for enabling the management of GLAs by the middleware itself.
It follows the MAPE (Monitoring, Analysis, Planning and Execution) principle.
Possible execution actions are moving cells between different cloud instances (migration), duplicating/splitting cells of the GLA (mitosis), or removing cells.

\begin{figure}[!ht]
\centering
	\includegraphics[width=\textwidth]{detailed_architecture.pdf}
	\caption{Detailed Architecture of GLAPP and the Infrastructure}
	\label{fig:detailed}
\end{figure}
\todo{Explain the separate parts - DONE BY: Dinesh}

\noindent\todo{''I like Fig2 better, but there is still plenty of room for improvement. For instance, I would remove the word "optimal" from the figure - especially, if we now use this iterative Q-learning approach, our deployment will often *not* be optimal, at least temporarily. I am also quite suspicious that the msot dominant block in the figure is the provisioner. Is this really what we want to emphasize?''}


%%%%%%%%%%%%%%%%%%%%%%%%%%%%%%%%%%%%%%%%%%%%%%%%%%%%%%%%%
% NEW CHAPTER																						%
% Remember to always use a new line for a new sentence! %
%%%%%%%%%%%%%%%%%%%%%%%%%%%%%%%%%%%%%%%%%%%%%%%%%%%%%%%%%
\chapter{10 PAGES: Components}

%%%%%%%%%%%%%%%%%%%%%%%%%%%%%%%%%%%%%%%%%%%%%%%%%%%%%%%%%
% NEW SECTION																						%
%%%%%%%%%%%%%%%%%%%%%%%%%%%%%%%%%%%%%%%%%%%%%%%%%%%%%%%%%
\section{3 PAGES: Backend}
\todo{DONE BY: Fabio}


%%%%%%%%%%%%%%%%%%%%%%%%%%%%%%%%%%%%%%%%%%%%%%%%%%%%%%%%%
% NEW SECTION																						%
%%%%%%%%%%%%%%%%%%%%%%%%%%%%%%%%%%%%%%%%%%%%%%%%%%%%%%%%%
\section{2 PAGES: GUI}
\todo{DONE BY: Dinesh}


%%%%%%%%%%%%%%%%%%%%%%%%%%%%%%%%%%%%%%%%%%%%%%%%%%%%%%%%%
% NEW SECTION																						%
%%%%%%%%%%%%%%%%%%%%%%%%%%%%%%%%%%%%%%%%%%%%%%%%%%%%%%%%%
\section{4 PAGES: MAPE}
\todo{DONE BY: Adrian / Riccardo}

MAPE is the control loop of the platform and responsible for managing the deployment of applications in accordance to the policy.
It analyze the environment information and performance metrics to determine the healthiness of application and trigger appropriate adaptation action should there be any violation to the policy.
The environment information includes details of the infrastructure, application deployment and user-defined policy.

MAPE consists of three parts: one is interfacing systems to retrieve the necessary information for analysis and dispatch the adaptation action, another part is analyzing the application healthiness in regard to the policy, and the last part is determining the adaptation action based on the environment information and computed healthiness.

At the start of each control loop, MAPE interfaces with SAILS backend to retrieve environment information, and with monitoring system to retrieve performance metrics. On the other hand, MAPE communicates to SAILS backend to execute the adaptation action.
As mentioned in the implementation decision section, MAPE use HTTP protocol with JSON object for data exchange.
It retrieves the infrastructure, application deployment information and user-defined policy from SAILS backend as well as performance metrics through HTTP requests.
For instance, infrastructure information includes information of virtual machines such as the service provider, machine location and machine tier. Deployment information, on the other hand, refers to the information of each application components including on which containers they are currently deployed.
Furthermore, MAPE also retrieve user-defined policy through SAILS backend. A policy is set of rules that defines which metrics the application need to comply to, its threshold for healthiness determine and the weight of each rule relative to the others.
Lastly, MAPE retrieve the performance metrics from the backend monitoring system.
JSON objects are used for all these information retrieval.
The use of JSON objects provides a data structure that is flexible to represent different types of information from different interfacing systems.
In addition, it allows MAPE to build a object-oriented view of the whole environment that provide quick access to specific information in various computation during the analysis.

% computation of healthiness values
\todo{elaborate about healthiness computation}

After the healthiness values of the each individual cell is computed based on the policy and performance metric, all violating cells within an organ are taken into account for an evaluation.
When the number of violating cell exceed a defined threshold, the organ will be considered as unhealthy.
Once an organ is unhealthy, information of all violations will be further processed in MAPE to determine appropriate adaption action.

% burlap library for MDP and its extensibility
In the first implementation of MAPE, MDP is used to determine the adaptation action.
Before starting the MDP, the environment and application deployment need to be modelled as states. At the same, actions for state transition are defined.
BURLAP library and its framework is used for modelling and transition action definition.
When the state and state-action information is constructed, learning algorithm from the library is applied.
The learning algorithm is first applied in a simulated environment to verify correctness of the constructed model and defined transition action.
It also serve a second purpose for understanding MDP process by generating the log of MDP with the aim for fine tuning the parameters in later stage of development.
While MDP works close to the expected outcome in a simulated environment, concern was raised regarding the convergence of Q-value that learning algorithm rely on to determine the optimal action in a given state.
More importantly, problem of high number of states and transition actions affects the viability of applying MDP to the optimization problem of MAPE.
After subsequent iteration of adjustment and refinement to the model and transition actions and execution of MDP in the simulated environment, it is concluded that reduction of number of states is required for applying MDP.
To reduct the number of states, value function approximation need to be used.

% simulation using burlap



%%%%%%%%%%%%%%%%%%%%%%%%%%%%%%%%%%%%%%%%%%%%%%%%%%%%%%%%%
% NEW SECTION																						%
%%%%%%%%%%%%%%%%%%%%%%%%%%%%%%%%%%%%%%%%%%%%%%%%%%%%%%%%%
\section{2 PAGES: Other}
\subsection{Monitoring / Prometheus}
\todo{DONE BY: Adrian / Riccardo}

Prometheus \todo{add reference} is a monitoring system that can provide both infrastructure and application performance metrics.
Infrastructure metrics are provided by exporters, which is a set of library that expose metrics data of the host environments and the Docker containers.
Application metrics are provided with the use of custom-built program to expose the needed data in a Prometheus exposition format \todo{add reference}.
The use of Prometheus and its exposition format enable MAPE to access custom metrics such as application performance data or cost metrics of the host from various cloud service providers.

% range query and point query in Prometheus
Prometheus is also sophisticated in providing different types of query.
It supports range query and point query with additional functions that can be applied on the raw data when formulating a query.
This enables the fine-grained extraction of data that best fits the corresponding performance metric.
For instance, the rate function allows the direct retrieval of the rate of change in CPU time of a container with configurable data point range, data point interval and duration covered by each data point.
The robust query functionality provides two major advantages.
First of all, it reduces the complexity of computation logic of MAPE that is responsible for getting specific performance metric for subsequent healthiness computation.
Secondly, it allows quick tweaking of query parameters to obtain the most relevant data for adaptation action determination during later development stage. 

\subsection{Supporting Applications}
\todo{Infrastructure Scripts, Consul, Registrator, Voting-App, Proxy, metrics-server etc.}
\todo{DONE BY: Fabio}



%%%%%%%%%%%%%%%%%%%%%%%%%%%%%%%%%%%%%%%%%%%%%%%%%%%%%%%%%
% NEW CHAPTER																						%
% Remember to always use a new line for a new sentence! %
%%%%%%%%%%%%%%%%%%%%%%%%%%%%%%%%%%%%%%%%%%%%%%%%%%%%%%%%%
\chapter{7 PAGES: Case Study}
\todo{DONE BY: Riccardo}

\section{Explain Demo Application}
\begin{itemize}
	\item Explain concept of the app at high level
	\item Explain the components themselves
	\item Explain the docker-compose file
	\item Explain our custom metrics (costs, clicks)
	\item Explain modifications to the app (region switch button, POST request with click-origin to metrics-server with every click
\end{itemize}

\section{Scenarios}
\begin{itemize}
	\item Explain the scenarios
	\item Explain the triggers
	\item Explain how MAPE realizes a necessary adaptation
\end{itemize}


%%%%%%%%%%%%%%%%%%%%%%%%%%%%%%%%%%%%%%%%%%%%%%%%%%%%%%%%%
% NEW CHAPTER																						%
% Remember to always use a new line for a new sentence! %
%%%%%%%%%%%%%%%%%%%%%%%%%%%%%%%%%%%%%%%%%%%%%%%%%%%%%%%%%
\chapter{2 PAGES: Future Work}
\todo{DONE BY: Fabio / Adrian}



%%%%%%%%%%%%%%%%%%%%%%%%%%%%%%%%%%%%%%%%%%%%%%%%%%%%%%%%%
% NEW CHAPTER																						%
% Remember to always use a new line for a new sentence! %
%%%%%%%%%%%%%%%%%%%%%%%%%%%%%%%%%%%%%%%%%%%%%%%%%%%%%%%%%
\chapter{1 PAGE: Conclusion}


\bibliographystyle{alpha}
\bibliography{sources}

%%%%%%%%%%%%%%%%%%%%%%%%%%%%%%%%%%%%%%%%%%%%%%%%%%%%%%%%%
% NEW CHAPTER																						%
% Evt. APPENDIX
% Remember to always use a new line for a new sentence! %
%%%%%%%%%%%%%%%%%%%%%%%%%%%%%%%%%%%%%%%%%%%%%%%%%%%%%%%%%


\end{document}
